\LineSpacing{2}


\begin{thebibliography}{10}

\bibitem{Mannheim2006} P.~D.~Mannheim,~Prog.~Part.~Nucl.~Phys.~{\bf 56},~340~(2006). 

\bibitem{Mannheim2009} P.~D.~Mannheim,~{\it Comprehensive solution to the cosmological constant, zero-point energy, and quantum gravity problems}, September 2009 (arXiv:0909.0212 [hep-th]). 

\bibitem{Mannheim2010a} P.~D.~Mannheim,~{\it Intrinsically quantum-mechanical gravity and the cosmological constant problem}, May 2010 (arXiv:1005.5108 [hep-th]). 

\bibitem{footnote1} As well as being of interest here as a classical theory, in  C.~M.~Bender and P.~D.~Mannheim, Phys.~Rev.~Lett.~{\bf 100},~110402 (2008);~Phys.~Rev.~D {\bf 78},
025022 (2008) it has been shown that as a quantum theory the  fourth-order conformal gravity theory is  both consistent and unitary.


\bibitem{Mannheim1989} P.~D.~Mannheim and D.~Kazanas,~Ap.~J.~{\bf 342}, 635 (1989).

\bibitem{Mannheim1994} P.~D.~Mannheim and D.~Kazanas,~Gen.~Rel.~Gravit.~{\bf 26}, 337 (1994).

\bibitem{Mannheim1992} P.~D.~Mannheim,~Ap.~J.~{\bf 391}, 429 (1992). 


\bibitem{Mannheim1997} P.~D.~Mannheim,~Ap.~J.~{\bf 479}, 659 (1997).


\bibitem{Mannheim2010b1} P.~D.~Mannheim and J.~G.~O'Brien,~{\it Fitting galactic rotation curves with conformal gravity and a global quadratic potential}, in preparation.  In this paper we give full references to the data sets we use. In the fits we have included the effects of bulges in NGC 5371, NGC 2998, NGC 5055, NGC 5033, NGC 801, NGC 5907, NGC 5533, NGC 6674 and UGC 2885, but not the effects of a bar in NGC 6674 or lopsidedness of the bulge in NGC 5533. While the inner region rotation curve is sensitive to a disk-bulge-bar decomposition of the luminosity, only the net luminous mass appears in the asymptotic (\ref{E21}), with the outer region cancellation effected by the $-\kappa c^2R$ term not depending on how much of the net $N^*$ is disk, bulge or bar.


\bibitem{footnote2} In \cite{Mannheim2010b} we give full references to the data sets we use. 

\bibitem{footnote3} We have included the effects of bulges in NGC 2998, NGC 5033, NGC 801, NGC 5533, UGC 2885 and NGC 6674,  but have not included bars in NGC 5533 and NGC 6674. While the inner region rotation curve is sensitive to a disk-bulge-bar decomposition of the luminosity, only the net luminous mass appears in the asymptotic (\ref{E21}), with the outer region cancellation effected by the $-\kappa c^2R$ term  thus not depending on how much of the net $N^*$ is disk, bulge or bar.


\bibitem{Milgrom1983} M.~Milgrom,~Ap.~J.~ {\bf 270}, 365, 371, 384 (1983).

\bibitem{Brownstein2006} J.~R.~Brownstein and  J.~W.~Moffat,~Ap.~J.~{\bf 636}, 721 (2006).

\bibitem{footnote4}  That these various alternate theories all work is because they each possess an underlying universal structure, with the last column in Table (1) and the data presented in
\cite{Mannheim2010b1} indicating that in each galaxy the centripetal accelerations measured at the last data points possess it too. For the moment such universality is not explained by dark matter theory.


%******************************************************************************************************************************************%


\bibitem{arfken} Arfken and Webber, Mathematical Methods for Physicists, 6th edition, Academic press, 2005.
\bibitem{zwicky} Zwicky, wikipedia.
\bibitem{rubin} V.~C.~Rubin,~W.~K.~Ford, and N.~Thonnard,~Ap.~J.~ {\bf 238},  471 (1980).
\bibitem{cas} C.~Carignan, R.~Sancisi and  T.~S.~van Albada, A.~J.~{\bf 95}, 37 (1988).

%(***)

%\bibitem{Mannheim2006} P.~D.~Mannheim,~Prog.~Part.~Nucl.~Phys.~{\bf 56},~340~(2006). 

%\bibitem{Mannheim2009} P.~D.~Mannheim,~{\it Comprehensive solution to the cosmological constant, zero-point energy, and quantum gravity problems}, September 2009 (arXiv:0909.0212 [hep-th]). 

%\bibitem{Mannheim2010a} P.~D.~Mannheim,~{\it Intrinsically quantum-mechanical gravity and the cosmological constant problem}, May 2010 (arXiv:1005.5108 [hep-th]). 

%\bibitem{Mannheim1989} P.~D.~Mannheim and D.~Kazanas,~Ap.~J.~{\bf 342}, 635 (1989).

%\bibitem{Mannheim1994} P.~D.~Mannheim and D.~Kazanas,~Gen.~Rel.~Gravit.~{\bf 26}, 337 (1994).

%\bibitem{Mannheim1992} P.~D.~Mannheim,~Ap.~J.~{\bf 391}, 429 (1992). 




\bibitem{Vareschi2010} G.~U.~Varieschi, Gen.~Rel. Gravit,~{\bf 42}, 929 (2010).


%\bibitem{Mannheim1997} P.~D.~Mannheim,~Ap.~J.~{\bf 479}, 659 (1997).

\bibitem{Mannheim2010b} P.~D.~Mannheim and J.~G.~O'Brien,~{\it Impact of a global quadratic potential on galactic rotation curves}, July 2010 (arXiv:1007.0970v1  [astro-ph.CO]). 

\bibitem{Walter2008} F.~Walter, E.~Brinks, W.~J.~G.~de Blok, F.~Bigiel, R.~C.~Kennicutt, M.~ D.~Thornley, and A.~Leroy, A.~J.~{\bf 136},  2563 (2008).

\bibitem{deBlok2008} W.~J.~G.~de Blok, F.~Walter, E.~Brinks, C.~Trachternach, S-H.~Oh, and R.~C.~Kennicutt,~A.~J.~{\bf 136}, 2648 (2008).

\bibitem{Verheijen1997} M.~A.~W.~Verheijen, Ph.~D.~Dissertation, Rijksuniversiteit Groningen (1997).

\bibitem{Leroy2008} A.~K.~Leroy, F.~Walter, E.~Brinks,  F.~Bigiel, W.~J.~G.~de Blok, B.~Madore, and M.~ D.~Thornley, A.~J.~{\bf 136},  2782 (2008).

\bibitem{Verheijen2001} M.~A.~W.~Verheijen, Ap.~J.~{\bf 563}, 694 (2001).

\bibitem{Verheijen2001b} M.~A.~W.~Verheijen and R.~ Sancisi, A.~A.~{\bf  370}, 765 (2001).

\bibitem{deBlok1997} W.~J.~G.~de Blok and S.~S.~McGaugh, Mon.~Not.~R.~Astron.~Soc.~{\bf 290}, 533 (1997).
\bibitem{deBlok1998} W.~J.~G.~de Blok and S.~S.~McGaugh, Ap.~J.~{\bf 508}, 132 (1998).



\bibitem{Kim2007} J.~H.~Kim, Ph.~D.~Dissertation, University of Maryland, 2007. 


\bibitem{Naray2006} R.~Kuzio de Naray, S.~S.~McGaugh, W.~J~G.~de Blok, and A.~Bosma,  Ap.~J.~Suppl.~Ser.~{\bf 165}, 461 (2006).



\bibitem{Naray2008} R.~Kuzio de Naray, S.~S.~McGaugh, and W.~J~G.~de Blok, Ap.~J.~{\bf  676}, 920 (2008).

\bibitem{Verheijen1999} M.~A.~W.~ Verheijen and  W.~J.~G.~de Blok, As.~Sp.~Sc.~{\bf 269-270}, 673 (1999).

\bibitem{deBlok2001} W.~J.~G.~de Blok, S.~S.~McGaugh and V.~C.~Rubin, A.~J.~{\bf 121}, 2381 (2001); {\bf 122}, 2396 (2001).
\bibitem{Hulst1993} J.~M.~van der Hulst, E.~D.~Skillman, T.~R.~Smith, G.~D.~Bothun, S.~S.~McGaugh, and W.~J.~G.~de Blok, A.~J.~{\bf 106}, 548 (1993).

\bibitem{deBlok1996} W.~J.~G.~de Blok and S.~S.~McGaugh, Ap.~J.~{\bf 469}, L89 (1996).



\bibitem{deBlok2001} W.~J.~G.~de Blok, S.~S.~McGaugh, A.~Bosma, and V.~C.~ Rubin, Ap.~J.~{\bf 552}, L23 (2001).


\bibitem{Mannheim1990} P.~D.~Mannheim,~Gen.~Rel.~Gravit.~{\bf 22}, 289 (1990).


\bibitem{Rubin1978} V.~C.~Rubin,~W.~K.~Ford, and N.~Thonnard,~Ap.~J.~ {\bf 238},  471 (1980).

\bibitem{Kent1986} S.~M.~Kent,~A.~J.~ {\bf 91}, 1301 (1986).

\bibitem{Broeils1992} A.~H.~Broeils,  Ph.~D.~Dissertation, Rijksuniversiteit Groningen (1992).

%\bibitem{Milgrom1983} M.~Milgrom,~Ap.~J.~ {\bf 270}, 365, 371, 384 (1983).



\bibitem{Moffat2005} J.~W.~Moffat, J.~Cos.~Ast.~Phys.~{\bf 05},~003 (2005);~{\bf 03},~004 (2006).

\bibitem{Sanders2002} R.~H.~Sanders and S.~S.~McGaugh, Ann.~Rev.~Astron.~Ap. {\bf 40}, 263 (2002).


\bibitem{Brownstein2006} J.~R.~Brownstein and  J.~W.~Moffat,~Ap.~J.~{\bf 636}, 721 (2006).

\bibitem{Navarro1996} J. F. Navarro, C. S. Frenk, and S. D. M. White,~Ap.~J.~{\bf 462}, 563 (1996); ibid. {\bf 490}, 493 (1997).


\bibitem{Andredakis1994} Y.~C.~Andredakis and R.~H.~Sanders,~Mon.~Not.~R.~Astron.~Soc.~{\bf 267}, 283 (1994).

\bibitem{Mannheim1996} P.~D.~Mannheim,~{\it Linear potentials in galaxies and clusters of galaxies},~astro-ph/9504022,~April~1995.

\bibitem{refa} G.~Lake, R.~A.~Schommer and J.~H.~van Gorkom, A.~J.~{\bf 99}, 547 (1990).

\bibitem{refb} M.-H.~Rhee, Ph.~D.~Dissertation, Rijksuniversiteit Groningen (1996).

\bibitem{refc} D.~Puche, C.~Carignan and R.~J.~Wainscoat,  A.~J.~{\bf 101},
447 (1991).

\bibitem{refd} C.~Carignan and D. Puche, A.~J.~{\bf 100}, 641 (1990).

\bibitem{refe} D.~Puche, C.~Carignan and A.~Bosma,  A.~J.~{\bf 100}, 1468 (1990).



\bibitem{reff} S.~Casertano and J.~H.~van Gorkom,  A.~J. {\bf 101}, 1231 (1991).

\bibitem{refg} M.~Jobin and C.~Carrignan, A.~J.~{\bf 100}, 648 (1990).


\bibitem{refh} K.~G.~Begeman, Ph.~D.~Dissertation, Rijksuniversiteit Groningen (1987).

\bibitem{refi} S.~Cote, C.~Carignan and R.~Sancisi, A.~J.~ {\bf 102}, 904 (1991).

\bibitem{refj} D.~Barnaby and H.~A.~Thronson, A.~J.~{\bf 107}, 1717 (1994).

\bibitem{Kent1985} S.~M.~Kent, Ap.~J.~Suppl.~Ser.~{\bf 59}, 115 (1985).
\bibitem{Kent1996} S.~M.~Kent, A.~J.~ {\bf 91}, 1301 (1986).


\bibitem{refl} C.~Carignan, R.~Sancisi and  T.~S.~van Albada, A.~J.~{\bf 95}, 37 (1988).

\bibitem{refm} P.~R.~Roelfsema and R~J.~Allen, A.~A.~{\bf 146}, 213 (1985).

\bibitem{Sanders1996} R.~H.~Sanders, Ap.~J.~{\bf 473}, 117 (1996).

\bibitem{Sanders2002} R.~H.~Sanders and S.~S.~McGaugh, Ann.~Rev.~Astron.~Ap. {\bf 40}, 263 (2002).

\bibitem{Swaters2009} R.~A.~Swaters, R.~Sancisi, T.~S.~van Albada and J.~M.~van der Hulst, A.~A.~{\bf 493}, 871 (2009).


%***************************************************************************************************************************************%

\bibitem{mallettref}
	  Ciufolini, I. and Wheeler, J.A. \emph{Gravitation and Inertia,} Chapter 5. Princeton University Press.  1995.
	\bibitem{rM00}
		Mallett, Ronald L.  Weak Gravitational Field of the Electromagnetic
		Radiation in a Ring Laser.  \emph{Physics Letters A} 269 (2000) 214-217.
	\bibitem{nB57}
		Balazs, N.L.  Effect of a Gravitational Field, Due To a Rotating Body,
		on the Plane of Polarization of an Electromagnetic Wave.  \emph{Physical
		Review.}  Volume 11, Number 1.  1958.
	\bibitem{lLeL00}
		Landau, L.D. and E.M. Lifshitz.  \emph{The Classical Theory of Fields.}
		Fourth Revised English Edition.  Butterworth-Heinemann.  2000.
	\bibitem{chandra}
  Stelmakh, N., Lopez, J., Smolski, O. and Roychoudhuri, C., June 2000, 100W 50ps gain switched pulses from vertical- stack laser diode, \emph{Electronics Letters}, Vol. 36, No.12, pp 1022-24.
  \bibitem{sarychev}A. K. Sarychev and V. M. Shalaev, \textit{Electrodynamics of Metamaterials} (World Scientific, New York, NY, 2007).
%***************************************************************************************************************************************%
\bibitem{shelby}R. A. Shelby, D. R. Smith, and S. Schultz, Science \textbf{292}, 77 (2001).

\bibitem{padilla}W. J. Padilla, D. N. Basov, and D. R. Smith, Mater. Today \textbf{9}, 28 (2006).

\bibitem{pendry}J. B. Pendry, Phys. Rev. Lett. \textbf{85}, 3966 (2000).

\bibitem{fang}N. Fang \textit{et al}., Science \textbf{308}, 534 (2005).

\bibitem{leonhardt1}U. Leonhardt and T. Philbin, Prog. Opt. 53, 70 (2009).

\bibitem{gordon}W. Gordon, Ann. Phys. \textbf{377}, 421 (1923).

\bibitem{tamm}I. Y. Tamm, J. Russ. Phys.-Chem. Soc. \textbf{56}, 248 (1924).

\bibitem{plebanski}J. Plebanski, Phys. Rev. \textbf{118}, 1396 (1960).

\bibitem{chen}H. Chen and C. T. Chan, Appl. Phys. Lett. \textbf{90}, 1105 (2007).

\bibitem{rahm}M. Rahm, D. Schurig, D. Roberts, S. Cummer, D. Smith, and J. Pendry, Photonics Nanostruct. Fundam. Appl. \textbf{6}, 87 (2008).

\bibitem{schurig1}D. Schurig \textit{et al}., Science \textbf{314}, 977 (2006).

\bibitem{schurig2}D. Schurig, J. B. Pendry, and D. R. Smith, Opt. Express \textbf{14}, 9794 (2006).

\bibitem{leonhardt2}U. Leonhardt and T. Tyc, Science \textbf{323}, 110 (2009).

\bibitem{jamesfootnote}The fact that light rays travel along geodesics in curved space can be shown by substituting a scalar field in the Klein Gordon equation in the Eikonal limit, in which case one recovers the geodesic equation of motion \cite{ellis}. Thus, flat space, while sufficient, is not necessary for the validity of transformation optics.

\bibitem{ellis}G. F. R. Ellis, in \textit{Relativistic Cosmology: Proceedings of the International School of Physics ``Enrico Fermi,'' Course XLVII}, edited by B. K. Sachs (Academic Press, New York, NY, 1971).

\bibitem{jacob2}Z. Jacob and E. E. Narimanov, Opt. Express \textbf{16}, 4597 (2008).

\bibitem{cummer}S. A. Cummer, B. I. Popa, D. Schurig, D. R. Smith, and J. Pendry, Phys. Rev. E \textbf{74}, 036621 (2006).

\bibitem{ABS}R. Adler, M. Bazin and M. Schiffer, \textit{Introduction to General Relativity} (McGraw-Hill, New York, NY, 1975).

\bibitem{footnote}The curve parameter $s$ may not be used as a differentiation variable because $ds=0$ for null geodesics. Thus a new parameter, $q$, must be chosen such that the null vector $dx^\mu/dq$ preserves its length under parallel displacement. Using the variational principle with this parameter yields the standard equations of motion and determines $q$ up to a linear transformation of $s$ \cite{schrodinger,ABS}. Thus, throughout this paper, $\dot{x}^\mu$ represents a derivative with respect to the affine parameter, $q$.

\bibitem{schrodinger}E. Schr\"{o}dinger, \textit{Expanding Universes} (Cambridge University Press, London, 1956).

\bibitem{ABSnote}While conic section orbits are possible for the attractive $1/r$ potential in the Newtonian limit, the exact relativistic solution (the Schwarszchild solution) gives rise to a quadratic term that is responsible for additional effects, e.g., the precession of the perihelion of Mercury \cite{ABS}. In this paper, we seek to find the metric that gives rise to solutions that are \textit{exact} conic sections for \textit{photons}.

\bibitem{goldstein}H. Goldstein, C. Poole and J. Safko, \textit{Classical Mechanics}, 3rd ed. (Addison Wesley, San Francisco, CA, 2002).

\bibitem{eisenhart}L. P. Eisenhart, \textit{Riemannian Geometry}, 6th ed. (Princeton University Press, Princeton, NJ, 1966).

\bibitem{brill}D. R. Brill, in \textit{Relativity, Astrophysics and Cosmology: Proceedings of the Summer School}, edited by W. Israel (Reidel, Boston MA, 1973), p. 127.

\bibitem{footnote2}To be pedantic, the spatial part of $g_{ij}$ is not generally the submatrix of $g_{\alpha\beta}$, which we shall denote as $g^{sub}_{ij}$. Instead, one can use the properties of the metric tensor to show that $g_{ij}=g^{sub}_{ij}-(g_{0i}g_{0j})/g_{00}$ \cite{LL}. Because our metric is diagonal, the second term vanishes and the expression reduces to $g_{ij}=g^{sub}_{ij}$ after all.

\bibitem{LL}L. D. Landau and E. M. Lifshitz, \textit{The Classical Theory of Fields}, 4th ed. (Butterworth Heinemann, New York, N.Y. 1975).
\bibitem{R3a} G.~F.~R.~Ellis,~{\it Relativistic Cosmology}, in Proceedings of the International School of Physics ``Enrico Fermi", Course XLVII, 1969, B.~K.~Sachs, Editor, Academic Press, New~York,~N.~Y.~(1971).

%***************************************************************************************************************************************%
\bibitem{R3a} G.~F.~R.~Ellis,~{\it Relativistic Cosmology}, in Proceedings of the International School of Physics ``Enrico Fermi", Course XLVII, 1969, B.~K.~Sachs, Editor, Academic Press, New~York,~N.~Y.~(1971).

\bibitem{R4} P.~D.~Mannheim,~ Prog.~Part.~Nucl.~Phys.~{\bf 56},~340~(2006). 

\bibitem{Q1} This is not all that can happen when one goes to curved space, since the wave equation of the scalar field itself can change. For instance, general covariance does not forbid the presence in the action of a direct coupling term $(\xi/12)S^2R^{\alpha}_{\phantom{\alpha}\alpha}$ between the scalar field and the Ricci scalar, with the scalar field equation of motion then being modified into $\nabla^{\mu}\nabla_{\mu}S+(\xi/6) SR^{\alpha}_{\phantom{\alpha}\alpha}=0$ where $\xi$ is a constant.


\bibitem{Q2} S. Weinberg, {\it Gravitation and Cosmology:
Principles  and Applications of the General Theory of Relativity} 
(Wiley, New York, 1972)




\bibitem{Q3} The equivalence principle does not require that all effects associated with a gravitational field can be removed at a given point by a coordinate transformation, as any Riemann tensor dependent term can never be brought to zero by a coordinate transformation. Rather, the content of the equivalence principle is  that at any given point in a curved space the Christoffel symbol dependent contribution to the geodesic equation can be removed via a coordinate transformation. Specifically, since the Christoffel symbols are not coordinate tensors, at any given point they can be brought to zero via a general coordinate transformation. Similarly, the quantity $d^2x^{\lambda}/d\tau^2$ is not a coordinate tensor either. It is only the specific linear combination $d^2x^{\lambda}/d\tau^2+\Gamma^{\lambda}_{\mu\nu}(dx^{\mu}/d\tau)( dx^{\nu}/d\tau)$ with the two terms having this very specific relative weight that is a coordinate tensor, to thus enforce the equality of the gravitational and inertial masses in (\ref{D10}) and (\ref{D11}), even as these two equations contain Riemann tensor dependent terms.

\bibitem{R1} P.~D.~Mannheim~and~D.~Kazanas,~Gen.~Rel.~Gravit.~{\bf 20},~201~(1988).

\bibitem{R2} Y.~Deng~and~P.~D.~Mannheim,~Gen.~Rel.~Gravit.~{\bf 20},~969~(1988).

\bibitem{R3} Y.~Deng~and~P.~D.~Mannheim,~Astrophys.~Sp.~Sci.~ {\bf 135},~261~(1987).



\bibitem{R3b} In passing we note that while the energy-momentum tensor given in (\ref{E29a}) and (\ref{E30}) recovers the perfect fluid form when one takes its flat space limit (as it of course must), it does so not by having $\pi_{\mu\nu}$ be a geometric quantity that vanishes in flat space (as would be the case if $\pi_{\mu\nu}$ were, say,  to be built out of tensors constructed from the Riemann tensor), but rather by having its coefficient $q_f$ vanish in the limit. In the flat space limit the $\pi_{\mu\nu}$ term thus vanishes dynamically rather than kinematically.


\bibitem{R4a} With the vectors $n^1_{\mu}=U_{\mu}+\sin\alpha V_{\mu}$ and $n^2_{\mu}=U_{\mu}+\sin\beta W_{\mu}$ being timelike for $U_{\mu}=(H^{1/2},0,0,0)$, $V_{\mu}=(0,J^{1/2},0,0)$, $W_{\mu}=(0,0,\rho J^{1/2},0)$ and arbitrary angles $\alpha$ and $\beta$, and with $n^1_{\mu}n^1_{\nu}T^{\mu\nu}$ and $n^2_{\mu}n^2_{\nu}T^{\mu\nu}$ respectively evaluating to $\rho_f+\sin^2\alpha (p_f+2q_f)$ and $\rho_f+\sin^2\beta(p_f-q_f)$, for large enough positive or negative $q_f$, it might be possible to violate the weak energy condition $n_{\mu}n_{\nu}T^{\mu\nu} >0$ for some timelike $n_{\mu}$. For strong gravity systems then, by generating the appropriate $q_f$ term, it might be possible to evade the Hawking-Penrose singularity theorems for collapsing stars by relaxing one of the assumptions which goes into the proof. Moreover, we would note that since a condition such as the weak energy condition is initially motivated by familiarity with standard fluids in flat spacetime (where there is no $q_f$ term and the quantity $\rho_f+p_f$ is positive), its extension to curved space presupposes that the only role of gravity is to generalize a flat space condition to its covariant form, and not to act dynamically in a way that might prevent the covariant generalization from actually holding.

\bibitem{R5} For modes of the form $j_{\ell}(\omega r)P^m_{\ell}(\theta)$, periodic boundary conditions require that $\ell$ be even.

\bibitem{R6} P.~D.~Mannheim,~{\it Linear potentials in galaxies and clusters of galaxies},~astro-ph/9504022,~April~1995. 

\bibitem{R7} K.~Huang~{\it Statistical Mechanics},~Second~Edition,~J.~Wiley,~New~York,~N.~Y.~(1987).

\bibitem{R7a} In the theory of white dwarf stars the degeneracy pressure of the electrons in the star is ordinarily calculated by introducing plane wave momentum eigenstates and filling up the Fermi sea to a maximum momentum $k_F=\hbar(3\pi^2 n)^{1/3}$, where $n$ is the electron number density.  Since such momentum eigenstates are associated with spatial translation invariance, in order to be able to use them at all the number density $n$ and the pressure $p=(1/3\pi^2\hbar^3)\int_0^{k_F}dk k^4/(k^2+m_e^2)^{1/2}$ would have to be independent of position,  and thus not be able to provide the pressure gradient needed to balance the gravitational attraction of the nuclei in the star. To obtain the needed pressure gradient in a weak gravity star one must use not the equilibrium Maxwell-Boltzmann (or Fermi-Dirac) distribution $f_{\rm MB}({\bf x}, {\bf p},t)$ but rather a spatially dependent departure from it, viz. the most probable distribution $f_{\rm mp}({\bf x}, {\bf p},t)$ as evaluated with slowly varying number density, average velocity, and temperature. One thus considers the star to be divided into concentric shells, with there being no spatial variation within any specific shell (so that one can use plane wave momentum states within each such shell), but with each shell having a maximum momentum $k_F$ that depends on the radius of the shell. In this way the pressure becomes dependent on the radial dependence of the number density $n$, and thereby generates the needed pressure gradient; and one is thus able to use momentum eigenstates in a weak gravity star even though the electron number density depends on position. However, for a strong gravity star one is no longer free to use momentum eigenstates and restrict to slowly varying modifications to the Maxwell-Boltzmann distribution $f_{\rm MB}({\bf x}, {\bf p},t)$. Rather, to construct the partition function, one must explicitly solve the Dirac equation in the curved space background and use whatever solutions to the radial equation then emerge. As we noted in Sec.~\ref{s3}, the full calculation is a highly non-linear one in which one must construct the energy-momentum tensor by an incoherent averaging over the solutions to the curved space wave equation, and then have this very same energy-momentum tensor  serve as the source of the Einstein equations so as to fix the metric coefficients that are to be used to obtain the solutions to the curved space wave equation in the first place. 

\bibitem{Q4} For a normal star (viz. one far from any possible cold late time state) the interior temperature is quite high. With high temperatures favoring high frequency Boltzmann factors, again we see that in normal stars low frequency modes are suppressed.


\bibitem{R8} J.~Binney~and~S.~Tremaine,~{\it Galactic Dynamics},~Princeton~University~Press,~Princeton,~N.~J.~ (1987).



\bibitem{R9} As noted in \cite{R8}, by multiplying (\ref{E52}) by ${\bf x}\cdot{\bf v}$, integrating over all ${\bf v}$ and all ${\bf x}$, and dropping all asymptotic surface terms, one obtains $\int d^3{\bf x}\langle {\bf v}^2 \rangle-\int d^3{\bf x}\langle {\bf x}\cdot \partial V({\bf x})/\partial {\bf x}\rangle=\partial(\int d^3{\bf x} \langle{\bf x}\cdot{\bf v}\rangle)/\partial t+N^2\int d^3{\bf x}d^3{\bf v}d^3{\bf x}_2d^3{\bf v}_2g({\bf x},{\bf v},{\bf x}_2,{\bf v}_2,t){\bf x}\cdot\partial\phi({\bf x},{\bf x}_2)/\partial {\bf x}$, an expression which reduces to the familiar virial relation when there are no correlations.

\bibitem{R10} The use of the method of the most probable distribution as used above for stars is not readily applicable here, since for particles whose motions are controlled by long range forces alone (i.e. in the absence of short range forces), the cluster cannot readily be approximated by shells of particles that do not  exchange energy and momentum with particles in other shells.


%*******************************************************************************************************************************%

\end{thebibliography}
